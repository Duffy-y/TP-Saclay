\documentclass[10pt]{article}
\usepackage[left=2cm, top=3cm, right=2cm, bottom=3cm]{geometry}
\usepackage[utf8]{inputenc}      % accents dans le source
\usepackage[T1]{fontenc}
\usepackage[french]{babel}
\usepackage{graphicx}
\usepackage{graphics}
\usepackage{amsmath}
\usepackage{tikz}
\usepackage{graphicx}
\usepackage{xcolor}
\usepackage{parskip}


\title{\textbf{Instrumentation} \\ TP 1: Diviseur de tension}
\author{MENARD Alexandre \\ VIEILLEDENT Florent}

\setlength{\parindent}{1cm}

\begin{document}
\maketitle

\section*{Introduction}
Dans ce travail pratique, on s'intéressera à mesurer la tension aux bornes d'une résistance à l'aide de différents appareils 
comme un voltmètre numérique, un voltmètre analogique ainsi qu'un oscilloscope. En sachant que chacun de ses appareils de mesure
disposent d'une résistance interne, la mise en dérivation d'un tel appareil sur un circuit de deux résistances en séries reliées à 
un générateur constitue une montage de type diviseur de tension. On cherche donc à montrer la formule du pont diviseur de tension 
donnée par:
\begin{equation}
    U = \frac{R_2}{R_2 + R_1} * E
    \label{eqn:diviseur_tension}
\end{equation}

On notera $R_i$, la résistance interne de l'appareil de mesure, $R_2$, la résistance où l'on mesure la tension et $E$, la tension du 
générateur.

\section{Première expérience: résistances de $1k \Omega$}
\subsection{Montage}

On réalise le montage suivant qu'on alimente avec un générateur de tension $E=5V$ et deux résistances $R_1$ et $R_2$ avec $R_1=R_2=1k\Omega$.

\begin{figure}[!htbp]
    \begin{center}
        \rotatebox[origin=c]{-90}{\includegraphics[scale=0.2]{Schéma1.png}}
        \label{fig:schema1}
        \caption{Schéma de l'expérience 1}
    \end{center}
\end{figure}

\newpage

On souhaite mesurer le potentiel $V_A$, pour cela, on branche en dérivation l'appareil sur les bornes de $R_2$. Ainsi, 
le branchement commun (COM) est du même potentiel que la masse, que l'on note $V_B$, et la borne V sera de même potentiel 
que $V_A$. On a donc $U = V_A - V_B = V_A$. De cette manière, en mesurant $U$, on obtient $V_A$.

On mesure la tension $U$ au borne de $R_{2}$ à l'aide de différents appareils afin de mettre en évidence de potentielles différences
: un voltmètre analogique, un voltmètre numérique et un oscilloscope. On compile donc l'ensemble des tensions que l'on a mesuré dans
le tableau suivant:

\begin{table}[h!]
    \begin{center}
        \begin{tabular}{|c|c||c|c||c|}
            \hline
            Calibre & Voltmètre analogique & Calibre & Voltmère numérique & Oscilloscope \\
            \hline
            $3V$ & $2.5 \pm 0.25V$ & $20V$ & $2.504 \pm 0.03V$ & $2.56 \pm 0.04V$ \\
            $10V$ & $2.5 \pm 0.5V$ & $200V$ & $2.50 \pm 0.02V$ &  \\
            $30V$ & $2.5 \pm 0.25V$ & $2000V$ & $2.5 \pm 1V$ &  \\
            $100V$ & $3 \pm 0.5V$ & & & \\
            \hline
        \end{tabular}
        \caption{Tension mesurée avec chaque appareil}
        \label{table:table1}
    \end{center}
\end{table}

\subsection{Théorie}
On s'intéresse ici à developper une formule permettant de présenter les résultats théoriques auquels on doit s'attendre par l'expérience
et de comparer le modèle à la théorie. On a l'expression de la tension $U$ qui est:

\begin{equation}
    U = V_A - V_B = V_A
\end{equation}

Cette égalité vient du fait que $V_B$ est au même potentiel que la masse, c'est à dire $V_B = 0V$. D'où $U = V_A$. On peut
désormais déterminer une expression de $V_A$ en utilisant la formule (\ref{eqn:diviseur_tension}):

\begin{equation}
    V_A = U = \frac{R_2}{R_2 + R_1} * E
\end{equation}

On peut enfin en déduire une valeur théorique en faisant une application numérique en prenant $R_1 = R_2 = 1 k\Omega$ et $E = 5V$.

\begin{equation}
    V_A = \frac{1 \times 10^3}{1 \times 10^3 + 1 \times 10^3} * 5 = 2.5V
\end{equation}

On peut également se questionner sur les couples de résistances à choisir si l'on souhaite diviser la tension du générateur par $n$
On réutilise donc la formule (\ref{eqn:diviseur_tension}) et avec $R_2 = 1k\Omega$, $E = 5V$ et l'on cherche $R_1$ tel que:

\begin{equation}
    \begin{split}
        V_A = \frac{E}{n} & \Rightarrow \frac{R_2}{R_2 + R_1} * E = \frac{E}{n} \\
        & \Rightarrow \frac{R_2 + R_1}{R_2} = n \\
        & \Rightarrow R_1 = (n-1)R_2 \\
    \end{split}
\end{equation}

On fait l'application numérique pour diviser la tension du générateur par $n = 2, 10$ et $100$, en notant $R_1(n)$, la valeur de $R_2$
pour diviser par $n$ la tension:
\begin{equation}
    R_1(2) = R_2 = 1k \Omega \quad R_1(10) = 9R_1 = 9k \Omega \quad R_1(100) = 99 R_1 = 99 k\Omega
\end{equation}

Comme l'on branche en dérivation le voltmètre avec sa résistance $R_i$, on peut considérer que $R_2$ et $R_i$ forment une seule résistance équivalente
qu'on note $R_2^{'}$ avec $R_2^{'} = \frac{R_iR_2}{R_i + R_2}$. On a donc: 

\begin{equation}
    V_A = U = \frac{R_2^{'}}{R_2^{'} + R_1} E = \frac{R_iR_2}{R_i + R_2} \frac{1}{\frac{R_iR_2}{R_i + R_2} + R_1} E = \frac{R_iR_2}{R_iR_2 + R_1(R_i + R_2)} E 
\end{equation}

On en déduit que $R_i$ doit être très grande devant $R_2$ pour que la mesure par le voltmètre ne perturbe pas la valeur.

\subsection{Comparaison entre théorie et expérience}
Selon la théorie, la tension $U$ doit valoir $2.5V$. On remarque que dans le tableau (\ref{table:table1}), la valeur théorique se retrouve
dans les valeurs mesurées expérimentalement en prenant en compte les incertitudes.

\section{Première expérience: résistances de $1M \Omega$}
On réalise le même montage que l'expérience n°1, mais on remplace les résistances par des résistances $R_1$ et $R_2$ de $1M\Omega$.

\subsection{Mesures}
On mesure la tension $U$ aux bornes de $R_2$ toujours avec les mêmes appareils, et l'on compile les valeurs dans le tableau suivant:

\begin{table}[h!]
    \begin{center}
        \begin{tabular}{|c|c||c|c||c|}
            \hline
            Calibre & Voltmètre analogique & Calibre & Voltmère numérique & Oscilloscope \\
            \hline
            $3V$ & $0.27 \pm 0.25V$ & $20V$ & $2.39 \pm 0.01V$ & $1.74 \pm 0.06V$ \\
            $10V$ & $0.75 \pm 0.5V$ & $200V$ & $2.40 \pm 0.01V$ &  \\
            $30V$ & $1.5 \pm 0.25V$ & $2000V$ & $2.4 \pm 0.1V$ &  \\
            $100V$ & $2 \pm 0.5V$ & & & \\
            \hline
        \end{tabular}
        \caption{Tension mesurée avec chaque appareil}
        \label{table:table2}
    \end{center}

    \subsection{Theorie}
    On peut reprendre les résultats de la partie théorie de la première expérience, en modifiant $R_1$ et $R_2$. On fait donc 
    l'application numérique:

    \begin{equation}
        V_A = U = \frac{R_2}{R_2 + R_1} * E = \frac{1 \times 10^6}{1 \times 10^6 + 1 \times 10^6} * 5 = 2.5V
    \end{equation}

    On peut également déterminer une expression de la résistance interne:
    
    \begin{equation}
        \begin{split}
            U = \frac{R_2^{'}}{R_2^{'} + R_1} * E & \Rightarrow U R_2^{'} + U R_1 = E R_2^{'} \\
            & \Rightarrow U R_1 = R_2^{'} (E - U) \\
            & \Rightarrow \frac{U R_1}{E - U} = R_2^{'} \\
            & \Rightarrow \frac{E - U}{UR_1} = \frac{1}{R_2} + \frac{1}{R_i} \\
            & \Rightarrow \frac{R_2 (E - U) - U R_1}{UR_1R_2} = \frac{1}{R_i} \\
            & \Rightarrow R_i = \frac{U R_1 R_2}{R_2 E - U (R_1 + R_2)}
        \end{split}
        \label{eqn:ri}
    \end{equation}

    \subsection{Comparaison entre expérience et théorie}
    On remarque que les valeurs mesurées en prenant en compte les incertitudes n'atteignent pas la valeur théorique que l'on a calculé.
    Le paramètre que l'on a modifié depuis la première expérience est la valeur des résistances $R_1$ et $R_2$. 
    Nous avons conclut précèdemment que la résistance interne $R_i$ doit être très grande devant $R_2$ pour que la mesure soit impactée,
    et dans cette expérience, on remarque que les mesures sont éloignées de la valeur théorique. 
    On peut donc se questionner sur la valeur des résistances internes des appareils de mesure, et si elles sont suffisamment grande
    pour mesure la tension aux bornes de ces résistances avec une très grande valeur.

\end{table}

\end{document}