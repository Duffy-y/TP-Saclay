\documentclass[12pt]{article}
\usepackage[left=3cm, top=3cm, right=3cm, bottom=3cm]{geometry}
\usepackage[utf8]{inputenc}      % accents dans le source
\usepackage[T1]{fontenc}
\usepackage[french]{babel}
\usepackage{graphicx}
\usepackage{graphics}
\usepackage{amsmath}
\usepackage{tikz}
\usepackage{xcolor}
\usepackage{mathtools}
\usepackage{parskip}

\title{\textbf{Instrumentation} \\ TP1 :Diviseur de tension}
\author{MENARD Alexandre \\ VIEILLEDENT Florent}

\begin{document}
\maketitle

\section*{Introduction}
\section{Première expérience: Résistances de $1k\Omega$ }
\subsection{Montage}
On réalise le montage suivant avec $E=5V$ et $R_{1}=R_{2}=1k\Omega$.
\begin{figure}[!htbp]
\begin{center}


\includegraphics[width=10cm]{Schéma1.png}
\label{Schéma1}
\caption{Schéma de l'expérience 1}
\end{center}
\end{figure}\\
On mesure la tension $U$ au borne de $R_{2}$. On utilise différend voltmètre: un voltmètre analogique, un voltmètre numérique et un oscilloscope. 
\end{document}