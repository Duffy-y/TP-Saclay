\documentclass[12pt]{article}
\usepackage[left=2cm, top=2cm, right=2cm, bottom=2cm]{geometry}
\usepackage[utf8]{inputenc}
\usepackage[T1]{fontenc}
\usepackage[french]{babel}
\usepackage{graphicx}
\usepackage{graphics}
\usepackage{amsmath}
\usepackage{tikz}
\usepackage{graphicx}
\usepackage{xcolor}
\usepackage{parskip}
\usepackage{physics}


\title{\textbf{Méthodes expérimentales} \\ TP 2: Collisions}
\author{MENARD Alexandre \\ VIEILLEDENT Florent}

\setlength{\parindent}{1cm}

\begin{document}
\maketitle

\section*{Introduction}
Dans ce travail pratique, on cherche à vérifier la validité de la conservation de la quantité de mouvement 
ainsi que de l'énergie cinétique au cours d'une collision ainsi que mettre en évidence les limites de la conservation énergétique.
Pour cela, on produira des collisions supposées élastiques entre deux chariots avec des conditions initales différentes. On étudiera 
la position et la vitesse de ces derniers pour affirmer ou infirmer
les deux lois. On s'appuiera comme pour les derniers travaux pratiques de Python pour l'analyse et la modélisation de nos données.


\newpage
\section{Expérience}

Le but de l'expérience est d'étudier l'énergie cinétique et la quantité de mouvement de deux mobiles avant et après un choc élastique. 

Pour réaliser les collisions, on utilise un banc de mécanique à niveau, avec deux chariots. On souhaite étudier des collisions élastiques, les chariots sont donc équipés d'aimants, de tel sorte que les chariots ne se touchent pas pendant la choc. Cette configuration permet de limiter l'énergie perdue sous forme de chaleur. Les chariots sont propulsés à la main ou à l'aide d'un propulseur à ressort. 

Concernant la caméra, on la règle sur 30 images par seconde avec ouverture à 1/500s, et permet de capturer la position des deux mobiles au cours du temps.
Enfin, on conservera seulement 4 à 6 images avant et après le choc pour pouvoir négliger les frottements sur ce petit laps de temps. On fixe l'origine de notre repère sur un point entre nos deux mobiles. 

On réalise une première collision élastique en propulsant chaque chariot simultanément avec une vitesse initiale faible.

Pour la seconde collision élastique, un mobile $M1$ est initialement en mouvement et le second mobile $M2$ est immobile.



\section{Exploitation des mesures}
Pour nos mesures, l'usage de différents appareils de mesure induit des incertitudes sur nos valeurs. On a:
\begin{itemize}
    \item l'incertitude de la position: $\delta x = 0.001m$ dû au mètre pour la mesure (l'incertitude de détermination de la position par le logiciel est inconnu)
    \item l'incertitude sur le temps: $\delta t = 0.033s$, liée au nombre d'image par seconde. 
    \item l'incertutde sur la masse des chariots: $\delta m_1 =\delta m_2 = 0.0001 kg$, qui est l'incertitude de la balance utilisée. Les valeurs des masses des chariots sont $m1=0.5286\,kg$ et $m2=5282\, kg$.
\end{itemize}

On note respectivement X1 et X2 les positions des mobiles M1 et M2 avant le choc. On note de la même manière X1' et X2' la position des mobiles après le choc. 

\newpage
Après avoir réaliser les expériences, on synthétise les données dans les deux graphiques suivants:
\begin{figure}[h!]
	\begin{center}
		\includegraphics[scale=0.5]{GrapheX.png}
		\label{GrapheX1,...}
		 \caption{Positions des mobiles pour la première expérience(à gauche) et la deuxième expérience(à droite)}
	\end{center}
\end{figure}

On remarque que pour les deux expériences, les points ne se rejoignent pas: c'est parce que le logiciel ne repère pas le bout des mobiles mais un post-it sur le mobile.  

Ensuite, on s'intéresse à calculer la vitesse, l'énergie cinétique ainsi que la quantité de mouvement
pour chaque mobile.
Pour calculer la vitesse $v$ à chaque instant $t$, on utilise sur la formule suivante:
\begin{equation}
    v(t) = \frac{x(t + \delta t) - x(t)}{\delta t}
\end{equation}

On note $t_i, x_i$, l'instant et la position initiale et $t_f, x_f$, l'instant et la position finale. On utilise donc la méthode
des pentes extrêmes pour obtenir l'incertitude sur la vitesse de chaque mobile:
\begin{gather*}
    v_{max} = \frac{(x_f + \delta x) - (x_i - \delta x)}{(t_f - \delta t) - (t_i + \delta t)} \\
    v_{min} = \frac{(x_f - \delta x) - (x_i + \delta x)}{(t_f + \delta t) - (t_i - \delta t)} \\
    \delta v = \frac{v_{max} - v_{min}}{2} \quad v = \frac{v_{max} + v_{min}}{2}
\end{gather*}

L'énergie cinétique ainsi que son incertitude sont données par:
\begin{equation}
    E_{ci} = \frac{1}{2}m_iv_i^2 \quad \text{ et } \quad \delta E_{ci} = \frac{1}{2}v_i^2 \times \delta m_i + m_iv_i \times \delta v_i
\end{equation}

Enfin, la quantité de mouvement ainsi que son incertitude sont données par:
\begin{equation}
    p_i = m_iv_i \quad \text{ et } \quad \delta p_i = v_i\delta m_i + m_i\delta v_i
\end{equation}

On trace donc la vitesse, l'énergie cinétique et la quantité de mouvement sur un graphique pour chaque expérience:

\[
    \text{GRAPHIQUE VITESSE/ENERGIE CINETIQUE/QUANTITE DE MOUVEMENT}
\]

\newpage
Ensuite, on s'intéresse à la conservation de la quantité de mouvement et de l'énergie cinétique du système pour
vérifier les lois de conservation. On note donc $E_c^{tot}$ et $p^{tot}$, l'énergie cinétique et la quantité de mouvement du système. 
Pour les incertitudes, on les somme simplement car les grandeurs d'un mobile sont indépendantes de l'autre mobile. On a par exemple:

\begin{align}
    p^{tot} & = m_1v_1 + m_2v_2 \\
    \Rightarrow \delta p^{tot} & = v_1\delta m_1 + m_1\delta v_1 + v_2\delta m_2 + m_2\delta v_2 \\
    \Rightarrow \delta p^{tot} & = \delta p_1 + \delta p_2
\end{align}

La dérivée de $v_1m_1$ en fonction de $v_2$ ou $m_2$ est nulle. On retrouve exactement le même résultat pour l'énergie cinétique et la vitesse.

\end{document}
