\documentclass[12pt]{article}
\usepackage[left=2cm, top=2cm, right=2cm, bottom=2cm]{geometry}
\usepackage[utf8]{inputenc}
\usepackage[T1]{fontenc}
\usepackage[french]{babel}
\usepackage{graphicx}
\usepackage{graphics}
\usepackage{amsmath}
\usepackage{tikz}
\usepackage{graphicx}
\usepackage{xcolor}
\usepackage{parskip}
\usepackage{physics}


\title{\textbf{Méthodes expérimentales} \\ TP 2: Collisions}
\author{MENARD Alexandre \\ VIEILLEDENT Florent}

\setlength{\parindent}{1cm}

\begin{document}
\maketitle

\section*{Introduction}
Dans ce travail pratique, on cherche à vérifier la validité de la conservation de la quantité de mouvement 
ainsi que de l'énergie cinétique au cours d'une collision ainsi que mettre en évidence les limites de la conservation énergétique.
Pour cela, on produira des collisions supposées élastiques entre deux chariots avec des conditions initales différentes. On étudiera 
la position et la vitesse de ces derniers pour affirmer ou infirmer
les deux lois. On s'appuiera comme pour les derniers travaux pratiques de Python pour l'analyse et la modélisation de nos données.


\newpage
\section{Expérience}
Pour réaliser les collisions, on utilise un banc de mécanique à niveau, avec deux chariots. On écarte les deux mires 
d'un mètre puis l'on place un chariot au niveau de chaque mire. On replacera le système dans cet état pour chaque nouvelle collision.
Enfin, chaque collision doit se produire entre les deux mires. 

Concernant la caméra, on la règle sur 30 images par seconde avec ouverture à 1/500s, et permet de capturer la position des deux mobiles au cours du temps.
Enfin, on conservera seulement 4 à 6 images avant et après le choc pour pouvoir négliger les frottements sur ce petit laps de temps.

On réalise une première collision élastique en propulsant chaque chariot simultanément avec une vitesse initiale faible.

Pour la seconde collision élastique, on positionne un chariot au milieu des deux mires, et on propulse le second chariot avec une vitesse initiale faible.

\section{Exploitation des mesures}
Pour nos mesures, l'usage de différents appareils de mesure induit des incertitudes sur nos valeurs. On a:
\begin{itemize}
    \item l'incertitude de la position: $\delta x = 0.001m$ dû au mètre pour la mesure (l'incertitude de détermination de la position par le logiciel est inconnu)
    \item l'incertitude sur le temps: $\delta t = 0.033s$
    \item l'incertutde sur la masse des chariots: $\delta m_1 = VALEUR kg$ et $\delta m_2 = VALEUR kg$
\end{itemize}

Après avoir les expériences, on synthétise les données dans les deux graphiques suivants:

\[
    \text{GRAPHIQUE DE X1/X'1, X2/X'2 ICI}
\]
\newpage

Ensuite, on s'intéresse à calculer la vitesse, l'énergie cinétique ainsi que la quantité de mouvement
pour chaque mobile.
Pour calculer la vitesse $v$ à chaque instant $t$, on utilise sur la formule suivante:
\begin{equation}
    v(t) = \frac{x(t + \delta t) - x(t)}{\delta t}
\end{equation}

On note $t_i, x_i$, l'instant et la position initiale et $t_f, x_f$, l'instant et la position finale. On utilise donc la méthode
des pentes extrêmes pour obtenir l'incertitude sur la vitesse de chaque mobile:
\begin{gather*}
    v_{max} = \frac{(x_f + \delta x) - (x_i - \delta x)}{(t_f - \delta t) - (t_i + \delta t)} \\
    v_{min} = \frac{(x_f - \delta x) - (x_i + \delta x)}{(t_f + \delta t) - (t_i - \delta t)} \\
    \delta v = \frac{v_{max} - v_{min}}{2} \quad v = \frac{v_{max} + v_{min}}{2}
\end{gather*}

L'énergie cinétique ainsi que son incertitude sont données par:
\begin{equation}
    E_{ci} = \frac{1}{2}m_iv_i^2 \quad \text{ et } \quad \delta E_{ci} = \frac{1}{2}v_i^2 \times \delta m_i + m_iv_i \times \delta v_i
\end{equation}

Enfin, la quantité de mouvement ainsi que son incertitude sont données par:
\begin{equation}
    p_i = m_iv_i \quad \text{ et } \quad \delta p_i = v_i\delta m_i + m_i\delta v_i
\end{equation}

On trace donc la vitesse, l'énergie cinétique et la quantité de mouvement sur un graphique pour chaque expérience:

\[
    \text{GRAPHIQUE VITESSE/ENERGIE CINETIQUE/QUANTITE DE MOUVEMENT}
\]

\newpage
Ensuite, on s'intéresse à la conservation de la quantité de mouvement et de l'énergie cinétique du système pour
vérifier les lois de conservation. On note donc $E_c^{tot}$ et $p^{tot}$, l'énergie cinétique et la quantité de mouvement du système. 
Pour les incertitudes, on les somme simplement car les grandeurs d'un mobile sont indépendantes de l'autre mobile. On a par exemple:

\begin{align}
    p^{tot} & = m_1v_1 + m_2v_2 \\
    \Rightarrow \delta p^{tot} & = v_1\delta m_1 + m_1\delta v_1 + v_2\delta m_2 + m_2\delta v_2 \\
    \Rightarrow \delta p^{tot} & = \delta p_1 + \delta p_2
\end{align}

La dérivée de $v_1m_1$ en fonction de $v_2$ ou $m_2$ est nulle. On retrouve exactement le même résultat pour l'énergie cinétique et la vitesse.

\end{document}
