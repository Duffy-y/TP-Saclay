\documentclass[12pt]{article}
\usepackage[left=3cm, top=1cm, right=3cm, bottom=3cm]{geometry}
\usepackage[utf8]{inputenc}      % accents dans le source
\usepackage[T1]{fontenc}
\usepackage[french]{babel}
\usepackage{graphicx}
\usepackage{graphics}
\usepackage{amsmath}
\usepackage{tikz}
\usepackage{xcolor} 
\usepackage{mathtools}
\usepackage{parskip}
\usepackage{subcaption}
\usepackage[export]{adjustbox}
\usepackage{chemist}
\usepackage{rotating}
\usepackage{hyperref}
\hypersetup{colorlinks=true,linkcolor=blue}

\title{\textbf{TP2 Chimie Inorganique} \\ Complexes Thermochromes}
\author{MENARD Alexandre \\ VIEILLEDENT Florent}

\begin{document}
\maketitle

\section{But}
Dans ce travail pratique nous synthétisons et étudions les propriétés de deux complexes thermochromes : le tétrachloronickelate de bis(diéthylammonium) $((C_2H_5)_2NH_2)_2[NiCl_4]$ et le tétrachlorocuprate de bis(diéthylammonium) $((C_2H_5)_2NH_2)_2[CuCl_4]$.

\section{Principe}
La géométrie d'un complexe peut dépendre de la température. 
Dans notre cas, les ligands chloro induisent une forte répulsion électronique entre ligands ce qui favorise une géométrie tétraédrique en absence de tout effet de l'environnement.
Le contre ion $(C_2H_5)_2NH_2$ peut cependant former des liaisons hydrogènes avec les ligands chloro, ce qui favorise d'autre géométrie que celle tétraédrique.
Ces liaisons hydrogènes n'étant possible qu'à basse température, la géométrie de nos complexes changent donc lorqu'on augmente la tempérautre.

Ce changement de géométrie modifie l'énergie de transition, notamment celles des transitions d-d responsables de la couleur des complexes.
Les longueurs d'ondes absorbées, et donc les couleurs des complexes, ne sont alors pas les mêmes selon la température.

\newpage

\section{Complexe  de $((C_2H_5)_2NH_2)_2[NiCl_4]$}
\subsection{Synthèse}
On synthétise le complexe avec la réaction suivante :
\begin{align}
    2((C_2H_5)_2NH_2)Cl_{(s)} + NiCl_{2(s)} \longrightarrow ((C_2H_5)_2NH_2)_2[NiCl_4]_{(s)}
\label{eq1:Premiere synthese}
\end{align}

On pèse les masses suivantes de réactifs :
\begin{table}[h!]
    \begin{center}
        \begin{tabular}{|c|c|c|c|}
            \hline
            Nom & Masse (en g) & Masse molaire & Nombre de moles \\
            \hline
            $((C_2H_5)_2NH_2)Cl$ & 1.10 & 10.,5 & $1.0\times 10^{-2}$ \\
            \hline
            $NiCl_2$ & 0.65 & 129.6 & $5.0 \times 10^{-2}$ \\
            \hline
        \end{tabular}
    \end{center}
    \caption{Table des masses de réactifs pour la première expérience}
    \label{tab1:masse1}
\end{table}

Les réactifs sont donc ajoutés en proportion steochiométrique dans cette expérience d'après l'équation (\ref{eq1:Premiere synthese}) et le tableau (\ref{tab1:masse1}).

1.10 g de $((C_2H_5)_2NH_2)Cl$ et 0.65 g de $NiCl_2$ sont mélangés dans un mortier pendant 10 minutes. 
$((C_2H_5)_2NH_2)Cl$ était sous la forme d'une poudre blanche et $NiCl_2$ une poudre orange.
La poudre orange obtenue est séchée pendant 1 heure dans une étude à $50^\circ C$.
Le solide est refroidi dans un dessicateur pendant 10 minutes. La poudre était alors jaune clair.

\subsection{Propriétés thermochromes}
Notre complexe a été placé sur un bance Kofler préalablement étalonné. 
Il a été déplacé jusqu'à l'observation d'un changment de couleur à une température $T_1=66\pm 1^\circ C$.
Le solide est alors passé du jaune clair à un bleu foncé.

À basse température, on a présence de liaisons hydrogènes entre $(C_2H_5)_2NH_2$ et les ligands chloro, ce qui favorise une géométrie octaédrique avec des chloro pontants.
La couleur observée est alors jaune claire, la couleur absorbée est alors la couleur complémentaire, le violet.
Cela correspond à une longueur d'onde absorbée $\lambda_1\approx 420 \ nm$. 
On peut calculer l'énergie de transition $\Delta E_1 =\frac{hc}{\lambda_1}=\frac{1237.8}{420}=2.95 \ eV$, avec $hc=1237.8 \ eV.nm$.

À haute température, on n'a plus de liaisons hydrogènes.
La géométrie favorisée est la géométrie tétraédrique qui limite la répulsion électronique entre les ligands chloro.
Le solide est alors bleu foncé, ce qui correspond à une couleur absorbée orange.
La longueur d'onde absorbée est alors $\lambda_2\approx 580 \ nm$, on a alors $\Delta E_2=2.13 eV$.

On a la relation $\Delta_t=\frac{4}{9}\Delta_O$, avec $\Delta_t$ la différence d'énergie entre les orbitales d dans un complexe tétraédrique et $\Delta_0$ la différence d'énergie dans un complexe octaédrique.
On s'attend donc à une diminution des énergies de transition lorsqu'on passe d'une géométrie octaédrique à une géométrie tétraédrique.
Nos résultats expérimentaux sont en accord avec les prédictions théoriques, on a $\Delta E_2 <  \Delta E_1$.

Dans cette expérience nous avons synthétisé le tétrachloronickelate de bis(diéthylammonium) et observé que le solide passait de jaune clair à bleu.
Nous avons expliqué ce changment de couleur par la diminution des énergies de transition d-d entre une géométrie octaédrique à basse température et une géométrie tétraédrique à haute température.

\section{Complexe  de $((C_2H_5)_2NH_2)_2[CuCl_4]$}
\subsection{Synthèse}
On synthétise le complexe avec la réaction suivante :
\begin{align}
    2((C_2H_5)_2NH_2)Cl_{(aq)} +  CuCl_{2(aq)} \longrightarrow ((C_2H_5)_2NH_2)_2[CuCl_4]_{(aq)}
\label{deuxiemesynthese}
\end{align}

On pèse nos réactifs :
\begin{table}[h!]
    \begin{center}
        \begin{tabular}{|c|c|c|c|}
            \hline
            Nom & Masse (en g) & Masse molaire & Nombre de moles \\
            \hline
            $((C_2H_5)_2NH_2)Cl$ & 1.10 & 10.,5 & $1.0\times 10^{-2}$ \\
            \hline
            $CuCl_2$ & 0.70 & 134.5 & $5.2 \times 10^{-2}$ \\
            \hline
        \end{tabular}
    \end{center}
    \caption{Table des masses de réactifs pour la deucxième expérience}
    \label{tab2:masse2}
\end{table}

D'après l'équation (\ref{deuxiemesynthese}) et le tableau (\ref{tab2:masse2}) on remarque que $((C_2H_5)_2NH_2)Cl$ est le réactif limitant.

Une première solution est préparée dans un bécher en ajoutant 1.10 g de chlorure de diéthylammonium $((C_2H_5)_2NH_2)Cl$ solide et 8 mL d'isopropanol.
0.70 g de chlorure de cuivre anhydre $CuCl_2$ et 2 mL d'éthanol absolue sont ajoutés dans un autre bécher. La solution de chlorure de diéthylammonium est incolore et celle de chlorure de cuivre anhydre est marron.
Les deux solutions sont chauffées à environ $40 ^\circ C$ dans un bain d'eau jusqu'à dissolution des solides.
La première solution est alors ajoutée dans le second bécher et la nouvelle solution est agitée à chaud pendant quelques minutes.
La solution est refroidie à température ambiante quelques minutes puis elle est placée dans un bain de glace.
Le complexe est précipité en frottant les parois du bécher avec un baguette de verre.
Le précipité vert clair est filtré sous vide avec un fritté puis lavé avec de l'isopropanol froid et de l'éther diéthylique froid.
Le résidu obtenu est alors séché quelques minutes à l'air libre. 

On obtient une masse de complexe $m_{exp}=0.87 \pm 0.05\ g$, l'incertitude prenant en compte que le solide est hygroscopique, ce qui limite le temps de filtration et de séchage du solide.
Cela correspond à un nombre de mol $n_{exp}=\frac{m_{exp}}{M_{complexe}}=\frac{0.87}{353.5}= 2.5 \ mmol$. 
Le nombre maximale de moles est $n_{theo}=\frac{n_{((C_2H_5)_2NH_2)Cl}}{\nu_{((C_2H_5)_2NH_2)Cl}}=5.0 \ mmol$ car le chlorure de diéthylammonium est le réactif limitant, avec $\nu_{((C_2H_5)_2NH_2)Cl}$ le coefficient steochiométrique.

Le rendement est donc $r=\frac{n_{exp}}{n_{theo}}=\frac{2.5}{5.0}=50 \%$. 
Pour notre calcul théorique, on a supposé la réaction totale, mais nous n'avons pas d'information sur la constante d'équilibre, ce qui peut expliquer un rendement de $50 \%$.
On note aussi que nous perdons du complexe lors de la filtration, le filtrat étant vert clair.
On perd du produit lors de l'étape de cristallisation, celle-ci étant faite en seulement quelques minutes.


\subsection{Propriétés thermochromes}

Le banc Kofler est utilisé pour étudier les propriétés thermochromes du complexe synthétisé. 
Un changement de couleur est observé à $T_2=55 \pm 2 ^\circ C$, le complexe passe de vert clair à orange. 
L'incertitude sur la température est plus grande qu'à la première expérience, le solide devient vite une pâte humide au contact de l'air, ce qui rend difficile la mesure sur le banc Kofler.

À basse température la géométrie attendue est plan carré, qui est stabilisée par les liasons hydrogènes entre les atomes de chlores et les ions diéthylammonium.
La couleur verte claire observée correspond à une couleur rouge absorbée.
On a donc $\lambda_3\approx 650 \ nm$ et $\Delta E_3 = 1.90 \ eV$.

À haute température la géométrie tétraédrique est stabilisée car il n'y a pas de liaisons hydrogènes, les interactions entre ligands chloro conduisent à la géométrie limitant la répulsion électronique.
La couleur observée est orange, la couleur absorbée est donc bleue. 
On en déduit que $\lambda_4\approx 430 \ nm$ et $\Delta E_4=2.88 \ eV$.

Les transitions d-d sont plus énergétiques dans un complexe plan carré que tétraédrique : supérieur ou égale à $\Delta_O$ pour le plan carré et égale à $\Delta_t$ pour la géométrie tétraédrique.
On peut donc s'attendre à $\Delta E_4 < \Delta E_3$, ce qui n'est pas le cas dans nos résultats expérimentaux.
La couleur du complexe à haute température est dû à un transfert de charge entre le métal et un ligand.
Cette transition est plus intense que les transitions d-d et absorbe vers 400 nm, ce qui est en accord avec nos observations.

Dans cette expérience nous avons synthétisé le  tétrachlorocuprate de bis(diéthylammonium). Nous avons observé que le solide passait de vert clair à orange pour une température de $55 ^\circ C$.
Nous avons remarqué que l'évolution du changement de couleur n'était pas expliquable par le changement de l'énergie de transition d-d, mais que la couleur du complexe à haute température était dû à un transfert de charge.

\section{Conclusion}

Dans ce travail pratique nous avons synthétisé deux complexes thermochromes.
Nous avons déterminé leur température de transition qui sont respectivement de $66 ^\circ C$ et $55 ^\circ C$ pour $((C_2H_5)_2NH_2)_2[NiCl_4]$ et le  $((C_2H_5)_2NH_2)_2[CuCl_4]$.
Nous avons expliqué ces changements de couleur par un changement de géométrie causé par la disparition de liaisons hydrogènes à haute température.
L'augmentation de la longeur d'onde absorbée pour le premier complexe est expliquée par la diminution de l'énergie de transition entre les orbitales d lorqu'on passe d'une géométrie octaédrique à une géométrie tétraédrique.
Nous avons observé que la couleur du second complexe était causée par un transfert de charge qui absorbait autour de 400 nm, et qu'on ne pouvait donc pas appliquer le même raisonnmenet que pour le premier complexe. 


\end{document}