\documentclass[12pt]{article}
\usepackage[left=2cm, top=1.5cm, right=2cm, bottom=1.5cm]{geometry}
\usepackage[utf8]{inputenc}
\usepackage[T1]{fontenc}
\usepackage[french]{babel}
\usepackage{graphicx}
\usepackage{graphics}
\usepackage{amsmath}
\usepackage{tikz}
\usepackage{xcolor} 
\usepackage{mathtools}
\usepackage{parskip}
\usepackage{subcaption}
\usepackage[export]{adjustbox}

\title{\vspace{-2cm}\textbf{TP 4 - Mesure du moment d'inertie d'un solide}}
\author{\vspace{-0.5cm}MENARD Alexandre - VIEILLEDENT Florent}
% \setlength{\parindent}{1cm}
\date{\vspace{-0.7cm}}


\newcommand{\ut}{\vec{u_\theta}}
\newcommand{\ur}{\vec{u_r}}
\newcommand{\uz}{\vec{u_z}}

\begin{document}
\maketitle

La mécanique des milieux continus (solides déformables et fluides) trouvent une forte application dans les milieux industriels notamment automobile afin de produire
des véhicules toujours plus sûr face aux risques d'accidents, et qui consomment le moins possible en optimisant l'écoulemement de l'air. Les avancées permettent aujourd'hui
de produire des véhicules atteignant des vitesses presque équivalentes à des avions de ligne. Ici, nous étudierons une branche plus idéalisé de la mécanique des milieux continus à savoir la mécanique des solides
indéformables, applicable sur des solides simples. Notre travail consistera alors à déterminer le moment d'inertie de plusieurs solides grâce à la période d'oscillation de ces derniers. Au préalable,
nous mesurerons le couple généré par le ressort ainsi que la constante de torsion du ressort.


\section{Mesure des grandeurs caractéristiques}
\subsection{Théorie}
On définit un répère polaire de centre $O$ confondu avec l'axe, et on note $\theta=0 \text{ rad}$ la position d'équilibre. On repère le point
de la mesure en $M$, tel que $\vec{OM} = r\ur$. En admettant que la force de rappel
s'applique orthoradialement avec $\vec{F} = F(r) \ut$, on exprime alors le moment $\vec{M}$ de $\vec{F}$:

\begin{align*}
    \vec{M} & = \vec{OM} \wedge \vec{F} & \vec{M} \cdot \uz = -D\theta \\
            & = -rF(r) \uz
\end{align*}

A priori, la formule précédente laisse sugérer que le couple est dépendant de $r$, mais la seconde dépendant de $\theta$
s'oppose à cette dépendance ($D$ étant constant en $kg.m^2.s^{-2}$). Nous vérifierons alors que le produit $rF(r)$ donne toujours la même valeur à $\theta$ fixe.
Enfin, à l'aide du théorème du moment cinétique, on obtient:
\begin{equation}
    \ddot \theta + \frac{D}{I}\theta = 0
\end{equation}

On en déduit alors la période en posant $\omega^2 = \frac{D}{I}$:
\begin{equation}
    T = \frac{2\pi}{\omega} = 2\pi \sqrt{\frac{I}{D}}
\end{equation}


\subsection{Mesure du couple}
Pour mesurer le couple, nous fixons le centre d'une barre sur l'axe, puis l'on fixe le dynamomètre à une distance $r$
de l'axe de rotation. On tire le dynamomètre pour modifier l'angle $\theta$ et obtenir une mesure de la force de rappel $\vec{F}$.
On s'assure également que le dynamomètre est horizontal et perpendiculaire à la barre, car dans le cas contraire, le ressort du dynamomètre
devra produire une force plus grande que nécessaire pour compenser la composante orthoradiale de $\vec{F}$.

En sachant que $\vec{F_{dyn}}$ et $\vec{F'_{dyn}}$ sont les forces mesurées par le dynamomètre (respectivement à l'horizontal et incliné), on a entre les deux cas:
\begin{equation}
    \left\lVert \vec{F} \right\rVert = \left\lVert \vec{F_{dyn}} \right\rVert < \left\lVert \vec{F'_{dyn}} \right\rVert \Rightarrow \text{La mesure inclinée est incorrecte}
\end{equation}

\begin{figure}[!h]
    \begin{center}
        \resizebox{0.5\textwidth}{3cm}{
            

\tikzset{every picture/.style={line width=0.75pt}} %set default line width to 0.75pt        

\begin{tikzpicture}[x=0.75pt,y=0.75pt,yscale=-1,xscale=1]
%uncomment if require: \path (0,300); %set diagram left start at 0, and has height of 300

%Straight Lines [id:da16292622750860142] 
\draw    (140,140) -- (72,140) ;
\draw [shift={(70,140)}, rotate = 360] [color={rgb, 255:red, 0; green, 0; blue, 0 }  ][line width=0.75]    (10.93,-3.29) .. controls (6.95,-1.4) and (3.31,-0.3) .. (0,0) .. controls (3.31,0.3) and (6.95,1.4) .. (10.93,3.29)   ;
%Shape: Spring [id:dp12760656496651346] 
\draw   (140,140) .. controls (140.75,135) and (143.75,130) .. (149.75,130) .. controls (161.75,130) and (161.75,150) .. (155.75,150) .. controls (149.75,150) and (149.75,130) .. (161.75,130) .. controls (173.75,130) and (173.75,150) .. (167.75,150) .. controls (161.75,150) and (161.75,130) .. (173.75,130) .. controls (185.75,130) and (185.75,150) .. (179.75,150) .. controls (173.75,150) and (173.75,130) .. (185.75,130) .. controls (197.75,130) and (197.75,150) .. (191.75,150) .. controls (185.75,150) and (185.75,130) .. (197.75,130) .. controls (209.75,130) and (209.75,150) .. (203.75,150) .. controls (197.75,150) and (197.75,130) .. (209.75,130) .. controls (221.75,130) and (221.75,150) .. (215.75,150) .. controls (209.75,150) and (209.75,130) .. (221.75,130) .. controls (233.75,130) and (233.75,150) .. (227.75,150) .. controls (221.75,150) and (221.75,130) .. (233.75,130) .. controls (236.39,130) and (238.46,130.97) .. (240,132.49) ;
%Straight Lines [id:da4294419662187403] 
\draw [color={rgb, 255:red, 208; green, 2; blue, 27 }  ,draw opacity=1 ]   (140,110) -- (208,110) ;
\draw [shift={(210,110)}, rotate = 180] [color={rgb, 255:red, 208; green, 2; blue, 27 }  ,draw opacity=1 ][line width=0.75]    (10.93,-3.29) .. controls (6.95,-1.4) and (3.31,-0.3) .. (0,0) .. controls (3.31,0.3) and (6.95,1.4) .. (10.93,3.29)   ;
%Straight Lines [id:da9928409816186436] 
\draw    (370,140) -- (302,140) ;
\draw [shift={(300,140)}, rotate = 360] [color={rgb, 255:red, 0; green, 0; blue, 0 }  ][line width=0.75]    (10.93,-3.29) .. controls (6.95,-1.4) and (3.31,-0.3) .. (0,0) .. controls (3.31,0.3) and (6.95,1.4) .. (10.93,3.29)   ;
%Shape: Spring [id:dp196556047968647] 
\draw   (370.22,140.04) .. controls (366.95,136.24) and (365,131.11) .. (368.19,127.93) .. controls (374.55,121.57) and (388.69,135.71) .. (384.45,139.95) .. controls (380.21,144.19) and (366.06,130.05) .. (372.43,123.69) .. controls (378.79,117.32) and (392.93,131.47) .. (388.69,135.71) .. controls (384.45,139.95) and (370.31,125.81) .. (376.67,119.45) .. controls (383.03,113.08) and (397.18,127.22) .. (392.93,131.47) .. controls (388.69,135.71) and (374.55,121.57) .. (380.91,115.2) .. controls (387.28,108.84) and (401.42,122.98) .. (397.18,127.22) .. controls (392.93,131.47) and (378.79,117.32) .. (385.16,110.96) .. controls (391.52,104.6) and (405.66,118.74) .. (401.42,122.98) .. controls (397.18,127.22) and (383.03,113.08) .. (389.4,106.72) .. controls (395.76,100.35) and (409.9,114.5) .. (405.66,118.74) .. controls (401.42,122.98) and (387.28,108.84) .. (393.64,102.47) .. controls (400.01,96.11) and (414.15,110.25) .. (409.9,114.5) .. controls (405.66,118.74) and (391.52,104.6) .. (397.88,98.23) .. controls (404.25,91.87) and (418.39,106.01) .. (414.15,110.25) .. controls (409.9,114.5) and (395.76,100.35) .. (402.13,93.99) .. controls (408.49,87.63) and (422.63,101.77) .. (418.39,106.01) .. controls (414.15,110.25) and (400.01,96.11) .. (406.37,89.75) .. controls (412.73,83.38) and (426.88,97.52) .. (422.63,101.77) .. controls (418.39,106.01) and (404.25,91.87) .. (410.61,85.5) .. controls (416.98,79.14) and (431.12,93.28) .. (426.88,97.52) .. controls (422.63,101.77) and (408.49,87.63) .. (414.85,81.26) .. controls (421.22,74.9) and (435.36,89.04) .. (431.12,93.28) .. controls (426.88,97.52) and (412.73,83.38) .. (419.1,77.02) .. controls (425.46,70.65) and (439.6,84.8) .. (435.36,89.04) .. controls (431.12,93.28) and (416.98,79.14) .. (423.34,72.78) .. controls (429.7,66.41) and (443.85,80.55) .. (439.6,84.8) .. controls (435.36,89.04) and (421.22,74.9) .. (427.58,68.53) .. controls (433.95,62.17) and (448.09,76.31) .. (443.85,80.55) .. controls (439.6,84.8) and (425.46,70.65) .. (431.82,64.29) .. controls (432.59,63.52) and (433.48,63.05) .. (434.43,62.83) ;
%Straight Lines [id:da8495881655350583] 
\draw [color={rgb, 255:red, 208; green, 2; blue, 27 }  ,draw opacity=1 ]   (370,140) -- (439.23,66.46) ;
\draw [shift={(440.6,65)}, rotate = 133.27] [color={rgb, 255:red, 208; green, 2; blue, 27 }  ,draw opacity=1 ][line width=0.75]    (10.93,-3.29) .. controls (6.95,-1.4) and (3.31,-0.3) .. (0,0) .. controls (3.31,0.3) and (6.95,1.4) .. (10.93,3.29)   ;
%Straight Lines [id:da9352078250890943] 
\draw  [dash pattern={on 0.84pt off 2.51pt}]  (440.6,65) -- (440,140) ;
%Straight Lines [id:da8571463134658666] 
\draw [color={rgb, 255:red, 245; green, 166; blue, 35 }  ,draw opacity=1 ]   (370,140) -- (438,140) ;
\draw [shift={(440,140)}, rotate = 180] [color={rgb, 255:red, 245; green, 166; blue, 35 }  ,draw opacity=1 ][line width=0.75]    (10.93,-3.29) .. controls (6.95,-1.4) and (3.31,-0.3) .. (0,0) .. controls (3.31,0.3) and (6.95,1.4) .. (10.93,3.29)   ;

% Text Node
\draw (97,112.4) node [anchor=north west][inner sep=0.75pt]    {$\vec{F}$};
% Text Node
\draw (159,72.4) node [anchor=north west][inner sep=0.75pt]  [color={rgb, 255:red, 208; green, 2; blue, 27 }  ,opacity=1 ]  {$\overrightarrow{F_{dyn}}$};
% Text Node
\draw (327,112.4) node [anchor=north west][inner sep=0.75pt]    {$\vec{F}$};
% Text Node
\draw (441,42.4) node [anchor=north west][inner sep=0.75pt]  [color={rgb, 255:red, 208; green, 2; blue, 27 }  ,opacity=1 ]  {$\overrightarrow{F'_{dyn}}$};
% Text Node
\draw (381,144.4) node [anchor=north west][inner sep=0.75pt]  [color={rgb, 255:red, 245; green, 166; blue, 35 }  ,opacity=1 ]  {$\overrightarrow{F'_{dyn,\theta }}$};


\end{tikzpicture}

        }
    \end{center}
    \caption{Importance de l'alignement du dynamomètre}
    \label{fig:horizontal}
\end{figure}

On souhaite vérifier que $\left\lVert \vec{M} \right\rVert$ ne dépend pas de $r$. On se fixe une position $\theta = \dots$ puis l'on mesure la force de rappel pour différentes
valeurs de $r$. Nous obtenons alors

Nous pouvons alors en conclure que le couple (et donc $rF(r)$) est indépendant de $r$.

\subsection{Mesure de $D$ par méthode statique}

On souhaite mesurer la constante de torsion $D$ en mesurant la norme de la force exercée sur le dynamomètre 
pour une plage de valeurs de $\theta$ à une distance $r$.

\begin{equation}
    D = \frac{rF}{\theta}
\end{equation}

Pour vérifier cette loi, nous traçons alors le produit $rF(r)$ en fonction de $1/\theta$, et l'on doit obtenir une valeur constante. Pour déterminer $D$,
nous effectuons simplement le calcul sur chaque mesure, et l'on effectue la moyenne de l'ensemble des valeurs de $D$.

Nous constatons bien une valeur constante et l'on obtient finalement $D = \dots$.


\subsection{Mesure de $D$ par méthode dynamique}
On propose ensuite de mesurer le moment d'inertie $I_b$ de la barre ainsi que la constante de torsion $D$. 
On place alors deux masses de masse $m$ chacune de manière symétrique par rapport à l'axe de rotation que l'on assimile à des points matériels. On a:
\begin{equation}
    I_{tot} = I_{masse} + I_b = 2mr_{masse}^2 + I_b
\end{equation}

Avec la formule de la période, on a:
\begin{align*}
    \frac{T^2}{4\pi^2} = \frac{I_{tot}}{D} = ar_{masse}^2 + b, \quad \quad a = \frac{2m}{D}, b=\frac{I_b}{D}
\end{align*}

Par régression linéaire, on obtient $a$ et $b$, ce qui nous permet de remonter à $D$ et $I_b$ avec:
\begin{equation}
    D = \frac{2m}{a} \quad I_b = \frac{2mb}{a}
\end{equation}

Pour mesurer la période, on fait faire un demi-tour à la barre puis on la lâche. Avec un chronomètre, 
on mesure le temps nécessaire pour qu'elle revienne à son point de départ à l'aide d'un chronomètre. On répète la mesure plusieurs fois
afin d'obtenir une valeur moyenne de $T$ avec une incertitude.

\break

[GRAPHIQUE]

[ANALYSE DU GRAPHIQUE + VALEUR SUR $D$ et $I_b$ + COMPARAISON METHODE STATIQUE VS DYNAMIQUE]

\section{Moment d'inertie de plusieurs solides}
Dans cette dernière partie, on se propose de déterminer le moment d'inertie de différents solides via la période
de rotation, puis de comparer les valeurs à la théorie. Pour cela, nous avons besoin des expressions des différents moments d'inerties que l'on peut trouver à la fin de l'énoncé
du TP. Nous donnons ici un exemple de calcul du moment d'inertie théorique pour une boule de rayon $R$ et de masse volumique $\rho$ uniforme.
\begin{align}
    I_{boule}   & = \frac{2}{3} \int_{r=0}^{R}\int_{\theta=0}^{2\pi}\int_{\phi=0}^{\pi} \rho(r,\theta,\phi) r^4 \sin(\theta) dr d\theta d\phi \\
                & = \frac{4 \times 2\pi \rho(r,\theta,\phi) R^5}{5 \times 3} \\
                & = \frac{2mR^2}{5}
\end{align}

On commence par lister la masse et les dimensions de chacun des solides à l'aide d'une balance imposant une incertitude $\delta m = 0.1g$
et une règle pour les rayons du cylindre creux donnant une incertitude $\delta r = 0.1cm$. Pour déterminer le moment d'inertie, nous plaçons le solide à étudier sur l'axe, et lui fait faire un demi-tour puis on le lâche. A l'aide d'une porte optique,
nous mesurons la période de rotation. \\ 
Pour la constante de torsion $D$ et le moment d'inertie de la barre, nous utiliserons les valeurs déterminées par la méthode dynamique car plus précise que la méthode
statique.

Grâce à la formule de la période, on a:
\begin{equation}
    I_{solide} = I_{tot} - I_b \Rightarrow I_{solide} = \frac{DT^2}{4\pi^2} - I_b
\end{equation}

On synthétise l'ensemble des mesures et des valeurs déduites dans le tableau suivant:

\begin{table}[h!]
	\centering
	\begin{tabular}{||c c c c c c||} 
		\hline
		Solide          & Masse $m$  & Rayon $r$  & Période & $I_{solide,exp}$ & $I_{solide,th}$    \\
		\hline
        Sphère          &            &          &           &                  &                    \\
        Cylindre creux  &            &          &           &                  &                    \\
        Cylindre plein  &            &          &           &                  &                    \\
		\hline
	\end{tabular}
	\caption{Caractéristiques des solides à étudier}
	\label{table:1}
\end{table}

[CONCLUSION SUR LES VALEURS DU TABLEAU]

[CONCLUSION TP FINALE]


\end{document}