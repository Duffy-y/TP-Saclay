\documentclass[12pt]{article}
\usepackage[left=2cm, top=1.5cm, right=2cm, bottom=1.5cm]{geometry}
\usepackage[utf8]{inputenc}
\usepackage[T1]{fontenc}
\usepackage[french]{babel}
\usepackage{graphicx}
\usepackage{graphics}
\usepackage{amsmath}
\usepackage{tikz}
\usepackage{xcolor} 
\usepackage{mathtools}
\usepackage{parskip}
\usepackage{subcaption}
\usepackage[export]{adjustbox}

\title{\vspace{-2cm}\textbf{TP 4 - Mesure du moment d'inertie d'un solide}}
\author{\vspace{-0.5cm}MENARD Alexandre - VIEILLEDENT Florent}
% \setlength{\parindent}{1cm}
\date{\vspace{-0.7cm}}


\newcommand{\ut}{\vec{u_\theta}}
\newcommand{\ur}{\vec{u_r}}
\newcommand{\uz}{\vec{u_z}}

\begin{document}
\maketitle

La mécanique des milieux continus (solides déformables et fluides) trouvent une forte application dans les milieux industriels notamment automobile afin de produire
des véhicules toujours plus sûr face aux risques d'accidents, et qui consomment le moins possible en optimisant l'écoulemement de l'air. Les avancées permettent aujourd'hui
de produire des véhicules atteignant des vitesses presque équivalentes à des avions de ligne. Ici, nous étudierons une branche plus idéalisé de la mécanique des milieux continus à savoir la mécanique des solides
indéformables, applicable sur des solides simples. Notre travail consistera alors à déterminer le moment d'inertie de plusieurs solides grâce à la période d'oscillation de ces derniers. Au préalable,
nous mesurerons le couple généré par le ressort ainsi que la constante de torsion du ressort.

\section{Mesure des grandeurs caractéristiques}
\subsection{Théorie}
On définit un répère polaire de centre $O$ confondu avec l'axe, et on note $\theta=0 \text{ rad}$ la position d'équilibre. On repère le point
de la mesure en $M$, tel que $\vec{OM} = r\ur$. En admettant que la force de rappel
s'applique orthoradialement avec $\vec{F} = -F(r, \theta) \ut$, on exprime alors le moment $\vec{M}$ de $\vec{F}$:

\begin{align}
    \vec{M} & = \vec{OM} \wedge \vec{F} & \vec{M} \cdot \uz = -D\theta \\
            & = -rF(r, \theta) \uz \nonumber
    \label{eqn:moment_ressort}
\end{align}

A priori, la formule précédente laisse sugérer que le couple est dépendant de $r$, mais la seconde dépendant de $\theta$
s'oppose à cette dépendance ($D$ étant constant en $kg.m^2.s^{-2}$). Nous vérifierons alors que le produit $rF(r)$ donne toujours la même valeur à $\theta$ fixe (voir partie (\ref{couple_constant})).
Enfin, à l'aide du théorème du moment cinétique, on obtient:
\begin{equation}
    \label{eqn:equa_diff}
    \ddot \theta + \frac{D}{I}\theta = 0
\end{equation}

On en déduit alors la période en posant $\omega^2 = \frac{D}{I}$:
\begin{equation}
    T = \frac{2\pi}{\omega} = 2\pi \sqrt{\frac{I}{D}}
    \label{eqn:periode}
\end{equation}

\subsection{Mesure du couple}
\label{couple_constant}

Pour mesurer le couple à un angle $\theta$ fixe, on pose une potence à environ $\theta \approx 90^{\circ}$ de la position d'équilibre pour maintenir un angle
quasi constant entre les mesures sur des $r$ différents. On fixe le dynamomètre à distance $r$ sur la barre puis l'on tire jusqu'à ce que la barre heurte la potence.
On relâche progressivement le dynamomètre afin d'obtenir la force minimale pour toucher la potence (car si l'on tire trop, on mesurera la force de réaction de la potence en plus de celle du ressort spirale).
On relève alors une plage de force où la barre commence à se décoller ce qui nous donnera une valeur moyenne et une incertitude sur notre mesure. On relève la force en $r=5, 10, 15, 20, 25cm$.

\begin{figure}[!h]
    \begin{center}
        \resizebox{0.5\textwidth}{3cm}{
            

\tikzset{every picture/.style={line width=0.75pt}} %set default line width to 0.75pt        

\begin{tikzpicture}[x=0.75pt,y=0.75pt,yscale=-1,xscale=1]
%uncomment if require: \path (0,300); %set diagram left start at 0, and has height of 300

%Straight Lines [id:da16292622750860142] 
\draw    (140,140) -- (72,140) ;
\draw [shift={(70,140)}, rotate = 360] [color={rgb, 255:red, 0; green, 0; blue, 0 }  ][line width=0.75]    (10.93,-3.29) .. controls (6.95,-1.4) and (3.31,-0.3) .. (0,0) .. controls (3.31,0.3) and (6.95,1.4) .. (10.93,3.29)   ;
%Shape: Spring [id:dp12760656496651346] 
\draw   (140,140) .. controls (140.75,135) and (143.75,130) .. (149.75,130) .. controls (161.75,130) and (161.75,150) .. (155.75,150) .. controls (149.75,150) and (149.75,130) .. (161.75,130) .. controls (173.75,130) and (173.75,150) .. (167.75,150) .. controls (161.75,150) and (161.75,130) .. (173.75,130) .. controls (185.75,130) and (185.75,150) .. (179.75,150) .. controls (173.75,150) and (173.75,130) .. (185.75,130) .. controls (197.75,130) and (197.75,150) .. (191.75,150) .. controls (185.75,150) and (185.75,130) .. (197.75,130) .. controls (209.75,130) and (209.75,150) .. (203.75,150) .. controls (197.75,150) and (197.75,130) .. (209.75,130) .. controls (221.75,130) and (221.75,150) .. (215.75,150) .. controls (209.75,150) and (209.75,130) .. (221.75,130) .. controls (233.75,130) and (233.75,150) .. (227.75,150) .. controls (221.75,150) and (221.75,130) .. (233.75,130) .. controls (236.39,130) and (238.46,130.97) .. (240,132.49) ;
%Straight Lines [id:da4294419662187403] 
\draw [color={rgb, 255:red, 208; green, 2; blue, 27 }  ,draw opacity=1 ]   (140,110) -- (208,110) ;
\draw [shift={(210,110)}, rotate = 180] [color={rgb, 255:red, 208; green, 2; blue, 27 }  ,draw opacity=1 ][line width=0.75]    (10.93,-3.29) .. controls (6.95,-1.4) and (3.31,-0.3) .. (0,0) .. controls (3.31,0.3) and (6.95,1.4) .. (10.93,3.29)   ;
%Straight Lines [id:da9928409816186436] 
\draw    (370,140) -- (302,140) ;
\draw [shift={(300,140)}, rotate = 360] [color={rgb, 255:red, 0; green, 0; blue, 0 }  ][line width=0.75]    (10.93,-3.29) .. controls (6.95,-1.4) and (3.31,-0.3) .. (0,0) .. controls (3.31,0.3) and (6.95,1.4) .. (10.93,3.29)   ;
%Shape: Spring [id:dp196556047968647] 
\draw   (370.22,140.04) .. controls (366.95,136.24) and (365,131.11) .. (368.19,127.93) .. controls (374.55,121.57) and (388.69,135.71) .. (384.45,139.95) .. controls (380.21,144.19) and (366.06,130.05) .. (372.43,123.69) .. controls (378.79,117.32) and (392.93,131.47) .. (388.69,135.71) .. controls (384.45,139.95) and (370.31,125.81) .. (376.67,119.45) .. controls (383.03,113.08) and (397.18,127.22) .. (392.93,131.47) .. controls (388.69,135.71) and (374.55,121.57) .. (380.91,115.2) .. controls (387.28,108.84) and (401.42,122.98) .. (397.18,127.22) .. controls (392.93,131.47) and (378.79,117.32) .. (385.16,110.96) .. controls (391.52,104.6) and (405.66,118.74) .. (401.42,122.98) .. controls (397.18,127.22) and (383.03,113.08) .. (389.4,106.72) .. controls (395.76,100.35) and (409.9,114.5) .. (405.66,118.74) .. controls (401.42,122.98) and (387.28,108.84) .. (393.64,102.47) .. controls (400.01,96.11) and (414.15,110.25) .. (409.9,114.5) .. controls (405.66,118.74) and (391.52,104.6) .. (397.88,98.23) .. controls (404.25,91.87) and (418.39,106.01) .. (414.15,110.25) .. controls (409.9,114.5) and (395.76,100.35) .. (402.13,93.99) .. controls (408.49,87.63) and (422.63,101.77) .. (418.39,106.01) .. controls (414.15,110.25) and (400.01,96.11) .. (406.37,89.75) .. controls (412.73,83.38) and (426.88,97.52) .. (422.63,101.77) .. controls (418.39,106.01) and (404.25,91.87) .. (410.61,85.5) .. controls (416.98,79.14) and (431.12,93.28) .. (426.88,97.52) .. controls (422.63,101.77) and (408.49,87.63) .. (414.85,81.26) .. controls (421.22,74.9) and (435.36,89.04) .. (431.12,93.28) .. controls (426.88,97.52) and (412.73,83.38) .. (419.1,77.02) .. controls (425.46,70.65) and (439.6,84.8) .. (435.36,89.04) .. controls (431.12,93.28) and (416.98,79.14) .. (423.34,72.78) .. controls (429.7,66.41) and (443.85,80.55) .. (439.6,84.8) .. controls (435.36,89.04) and (421.22,74.9) .. (427.58,68.53) .. controls (433.95,62.17) and (448.09,76.31) .. (443.85,80.55) .. controls (439.6,84.8) and (425.46,70.65) .. (431.82,64.29) .. controls (432.59,63.52) and (433.48,63.05) .. (434.43,62.83) ;
%Straight Lines [id:da8495881655350583] 
\draw [color={rgb, 255:red, 208; green, 2; blue, 27 }  ,draw opacity=1 ]   (370,140) -- (439.23,66.46) ;
\draw [shift={(440.6,65)}, rotate = 133.27] [color={rgb, 255:red, 208; green, 2; blue, 27 }  ,draw opacity=1 ][line width=0.75]    (10.93,-3.29) .. controls (6.95,-1.4) and (3.31,-0.3) .. (0,0) .. controls (3.31,0.3) and (6.95,1.4) .. (10.93,3.29)   ;
%Straight Lines [id:da9352078250890943] 
\draw  [dash pattern={on 0.84pt off 2.51pt}]  (440.6,65) -- (440,140) ;
%Straight Lines [id:da8571463134658666] 
\draw [color={rgb, 255:red, 245; green, 166; blue, 35 }  ,draw opacity=1 ]   (370,140) -- (438,140) ;
\draw [shift={(440,140)}, rotate = 180] [color={rgb, 255:red, 245; green, 166; blue, 35 }  ,draw opacity=1 ][line width=0.75]    (10.93,-3.29) .. controls (6.95,-1.4) and (3.31,-0.3) .. (0,0) .. controls (3.31,0.3) and (6.95,1.4) .. (10.93,3.29)   ;

% Text Node
\draw (97,112.4) node [anchor=north west][inner sep=0.75pt]    {$\vec{F}$};
% Text Node
\draw (159,72.4) node [anchor=north west][inner sep=0.75pt]  [color={rgb, 255:red, 208; green, 2; blue, 27 }  ,opacity=1 ]  {$\overrightarrow{F_{dyn}}$};
% Text Node
\draw (327,112.4) node [anchor=north west][inner sep=0.75pt]    {$\vec{F}$};
% Text Node
\draw (441,42.4) node [anchor=north west][inner sep=0.75pt]  [color={rgb, 255:red, 208; green, 2; blue, 27 }  ,opacity=1 ]  {$\overrightarrow{F'_{dyn}}$};
% Text Node
\draw (381,144.4) node [anchor=north west][inner sep=0.75pt]  [color={rgb, 255:red, 245; green, 166; blue, 35 }  ,opacity=1 ]  {$\overrightarrow{F'_{dyn,\theta }}$};


\end{tikzpicture}

        }
    \end{center}
    \caption{Importance de l'alignement du dynamomètre}
    \label{fig:horizontal}
\end{figure}

Lors de la mesure, on s'assure également que le dynamomètre est horizontal et perpendiculaire à la barre, car dans le cas contraire, le ressort du dynamomètre
devra produire une force plus grande que nécessaire pour compenser la composante orthoradiale de $\vec{F}$ (voir figure \ref{fig:horizontal}).

En sachant que $\vec{F_{dyn}}$ et $\vec{F'_{dyn}}$ sont les forces mesurées par le dynamomètre (respectivement à l'horizontal et incliné), on a entre les deux cas:
\begin{equation}
    \left\lVert \vec{F} \right\rVert = \left\lVert \vec{F_{dyn}} \right\rVert < \left\lVert \vec{F'_{dyn}} \right\rVert \Rightarrow \text{La mesure inclinée est incorrecte}
\end{equation}

On résume l'ensemble des mesures expérimentales et des valeurs déduites dans le tableau (\ref{table:couple_r_variable}) suivant:

\begin{table}[h!]
	\centering
	\begin{tabular}{||c | c c c c c||} 
		\hline
		$r \pm 0.2$ (en cm) & $5$ & $10$ & $15$ & $20$ & $25$ \\
		\hline
        $F \pm 0.03$ (en N) & $0.89$ & $0.45$ & $0.33$ & $0.25$ & $0.20$ \\
		\hline
        $|| \vec{M} ||$ (en mN.m) & $46 \pm 3$ & $45 \pm 4$ & $50 \pm 5$ & $50 \pm 7$ & $50 \pm 8$ \\
        \hline
    \end{tabular}
	\caption{Couple du ressort pour différents rayons $r$}
	\label{table:couple_r_variable}
\end{table}

Nous pouvons conclure que à l'incertitude près, l'ensemble des couples obtenus se situent dans un même intervalle. Le couple $M = rF(r)$ 
est donc constant selon $r$ pour $\theta$ fixe. Cependant, il est important de noter que les mesures présentent un manque de précision majeur,
il faudrait que la barre présente des crans à des rayons $r$ précis afin de maintenir la position du crochet du dynamomètre. Nous n'avions d'ailleurs
aucun moyen d'assurer l'horizontalité et la perpendicularité avec la barre du dynamomètre à part le jugement de l'expérimentateur.

\subsection{Mesure de $D$ par méthode statique}
\label{D_statique}

On souhaite mesurer la constante de torsion $D$ en mesurant la norme de la force exercée sur le dynamomètre 
pour une plage de valeurs de $\theta$ à une distance $r=5 \pm 0.2cm$ qu'on conservera constante entre différents 
angles $\theta$ en marquant un repère avec un feutre sur la barre. Pour la mesure, un expérimentateur positionne un
rapporteur au dessus de la barre à l'équilibre, et le second tire ensuite la barre avec le dynamomètre jusqu'à un angle $\theta$. 
Nous estimons l'incertitude sur $\theta$ à environ $\delta \theta = 3^{\circ} = 0.5.10^{-2}$rad de part la lecture approximative et la position
du rapporteur qui peut légèrement varier.

\begin{equation}
    D = \frac{rF}{\theta}
\end{equation}

\textbf{Remarque:} Les calculs seront \textbf{toujours} effectués en radians, bien que le tableau (\ref{table:mesure_D_statique}) indique
les angles en degrés pour faciliter la compréhension de notre démarche.

\break

\begin{table}[h!]
	\centering
	\begin{tabular}{||c | c c c c c c c c||} 
		\hline
		$\theta \pm 3$ (en $^\circ$) & $10$ & $30$ & $60$ & $90$ & $120$ & $150$ & $180$ & $210$ \\
		\hline
        $F \pm 0.03$ (en N) & $0.08$ & $0.26$ & $0.58$ & $0.86$ & $1.08$ & $1.30$ & $1.62$ & $1.88$\\
		\hline
        $D$ (en mN.m) & $23 \pm 16$ & $27 \pm 7$ & $28 \pm 4$ & $27 \pm 3$ & $26 \pm 2$ & $25 \pm 2$ & $26 \pm 2$ & $26 \pm 2$\\
        \hline
    \end{tabular}
	\caption{Constante de torsion $D$ selon l'angle $\theta$}
	\label{table:mesure_D_statique}
\end{table}

Malgré le manque de précision dû au matériel présent, nous pouvons conclure que $D$ est bien indépendant de $\theta$ et du couple $M$ avec $D=26 \pm 2 \text{ mN.m}$. 
Les mêmes améliorations que pour l'expérience précédente permettrait d'obtenir une meilleure précision. 
On ajoute que la présence d'une platine indiquant les angles sous la barre permettrait une meilleure
lecture de l'angle, et donc une valeur de $D$ avec des incertitudes plus petites.


\subsection{Mesure de $D$ par méthode dynamique}
On propose ensuite de mesurer le moment d'inertie $I_b$ de la barre ainsi que la constante de torsion $D$ à l'aide de la période $T$
de la barre avec deux masses $m$ se situant symétriquement par rapport à l'axe de rotation sur la barre à une distance $r_{masse}$ variable de l'axe au centre de masse de ces dernières, on prendra le centre géométrique des cylindres
comme centre de masse. Il nous faut alors exprimer $T$ comme une fonction de $r_{masse}$ pour obtenir 
$I_b$ et $D$. On commence par exprimer le moment d'inertie total $I_{tot}$ du système:
\begin{equation}
    I_{tot} = I_{masse} + I_b = 2mr_{masse}^2 + I_b
\end{equation}

Avec la formule de la période (\ref{eqn:periode}), on en déduit:
\begin{align*}
    \frac{T^2}{4\pi^2} = \frac{I_{tot}}{D} = ar_{masse}^2 + b, \quad \quad a = \frac{2m}{D}, b=\frac{I_b}{D}
\end{align*}

Par régression linéaire, on obtient $a$ et $b$, ce qui nous permet de remonter à $D$ et $I_b$ avec:
\begin{equation}
    D = \frac{2m}{a} \quad I_b = \frac{2mb}{a}
\end{equation}

On installe les deux masses de masse $m = 212 \pm 0.1g$ (obtenue sur une balance) en $r_{masse}$ avec une incertitude $\delta r = 0.2cm$ donnée par la règle et sa légère instabilité.
Pour mesurer la période, le même expérimentateur fait faire un demi-tour approximatif\footnote{L'angle de départ n'influence pas la période, voire partie (\ref{section:periode})} à la barre,
puis lâche cette dernière en démarrant le chronomètre pour limiter les temps de réaction. Enfin, il coupe le chronomètre lorsque la barre revient à un angle maximal du même côté que son point de
départ\footnote{La présence de frottements impose une diminution de l'amplitude, il est donc normal qu'on n'observe pas un retour au point de départ, cependant, 
la période ne varie pas comme on le montre dans la partie (\ref{section:periode})}. 
On répète 10 fois la mesure de la période en relançant la barre et cela pour chaque distance $r_{masse}$ 
afin d'obtenir une valeur moyenne et une incertitude statistique. On reporte les périodes moyennes pour 
chaque $r_{masse}$ dans le graphique (\ref{fig:graphe_T}) et le tableau (\ref{table:mesure_D_dynamique}) ainsi que les détails de calcul en 
partie (\ref{section:D_dynamique}).

En conclusion, nous obtenons bien une droite comme prédit par la théorie, avec $D = 24.7 \pm 0.2 mN.m$, qui conserve l'ordre de grandeur trouvé en partie (\ref{D_statique}), cependant, la méthode
dynamique utilisant la période est bien plus précise car elle nous permet d'obtenir une valeur de $D$ précise au dixième contre une précision
à l'unité pour la méthode statique. De plus, elle nous permet de trouver le moment d'inertie de la barre de $I_b = 3.93 \pm 0.09g.m^2$. Notre mesure pourrait
encore gagner en précision en utilisant une caméra ou une porte optique pour mesurer plus précisement la période.

\break



\section{Moment d'inertie de plusieurs solides}
Dans cette dernière partie, on se propose de déterminer le moment d'inertie de différents solides via la période
de rotation, puis de comparer les valeurs à la théorie. Pour cela, nous avons besoin des expressions des différents moments d'inerties que l'on peut trouver à la fin de l'énoncé
du TP. Nous donnons un exemple de calcul pour une boule dans la partie (\ref{boule})

On commence par lister la masse et les dimensions de chacun des solides à l'aide d'une balance imposant une incertitude $\delta m = 0.1g$.
On mesure les différents rayons $r$ des solides avec une règle, on impose alors une incertitude sur les rayons de $\delta r = 0.2cm$. Cependant, pour la boule nous conserverons
la même incertitude mais nous utiliserons le rayon fourni car il aurait fallu un fil pour mesurer la circonférence de la boule.

Chaque solide présente un trait noir, que nous utiliserons comme repère pour mesurer plus facilement la période. Pour la mesurer,
un même expérimentateur fait tourner le solide d'un demi-tour, puis libère le solide et démarre le chronomètre en simultanée. 
Une fois que le repère effectue un aller-retour, l'expérimentateur arrête le chronomètre lorsque 
le repère atteint un angle maximal \footnote{Même remarque concernant l'angle que dans la partie précédente, voir partie (\ref{section:periode})}.
On répète la mesure de la période 10 fois par solide pour obtenir une valeur moyenne ainsi qu'une incertitude tenant compte du temps de réaction
de l'expérimentateur. On obtient la valeur de $I_{th}$ par la formule présente dans la table Méca 3.1 du poly et $I_{exp}$ issu de la formule (\ref{eqn:periode}) avec:

\begin{equation}
    I_{exp} = \frac{DT^2}{4\pi^2} 
\end{equation}

On synthétise l'ensemble des mesures et des valeurs déduites dans le tableau suivant\footnote{Précision sur les calculs d'incertitudes dans la partie (\ref{incertitude})}:

\begin{table}[h!]
	\centering
	\begin{tabular}{||c c c c c c c||} 
		\hline
		Solide          & $m \pm 0.1$ (g)  & $r \pm 0.2$ (cm)  & Période (s) & $I_{exp}$ ($g.m^2$)& $I_{th}$ ($g.m^2$) & $\gamma$ (\%)\\
		\hline
        Sphère          & $644.9$ & $7.0$ & $1.34 \pm 0.02$ & $1.12 \pm 0.03$ & $1.26 \pm 0.07$ & $0\%$ \\
        Cylindre creux  & $347.3$ & $\frac{r_e}{r_i} = \frac{5}{4.6}$ & $1.08 \pm 0.02$ & $0.72 \pm 0.03$ & $0.8 \pm 0.1$ & $71\%$ \\
        Cylindre plein  & $386.1$ & $4.9$ & $0.81 \pm 0.02$ & $0.41 \pm 0.02$ & $0.46 \pm 0.04$ & $40\%$ \\
		\hline
	\end{tabular}
	\caption{Caractéristiques des solides et résultats expérimentaux}
	\label{table:1}
\end{table}

En conclusion, nous pouvons voir que pour les deux cylindres, les valeurs expérimentales et théoriques sont en accord, et les recouvrements $\gamma$
nous permettent de conclure que la théorie prédit de façon satisfaisante pour le cylindre creux et avec une précision plus faible pour le cylindre plein.
Cependant, pour la sphère, les deux valeurs ne se croisent pas avec $\gamma = 0\%$, donc on ne retrouve pas la valeur théorique par l'expérience.

Les écarts à la théorie pourraient être dû à la présence de frottements mécaniques (roulements) peuvent réduire la période. L'incertitude sur la période est cohérente car de l'ordre 
du temps de réaction humain, mais il reste l'incertitude sur le moment où l'expérimentateur estime que le solide atteint l'angle maximal. Pour résoudre ce dernier problème,
l'usage d'une caméra pourrait résoudre ce problème en suivant un repère comme nous l'avons fait à l'oeil. On obtiendrai alors une incertitude plus petite, et une
valeur de la période plus précise. Pour la valeur théorique, l'usage de balance plus précise ainsi que des rayons avec une incertitude plus faible serait intéressant.

\section{Conclusion}

\break
\section{Annexes}
\subsection{Mesure de $D$ par méthode dynamique, résultats et graphiques}
\label{section:D_dynamique}
\begin{table}[h!]
	\centering
	\begin{tabular}{||c | c c c c c||} 
		\hline
		$r_{masse} \pm 0.2$ (en cm) & $7$ & $12$ & $17$ & $22$ & $27$ \\
		\hline
        $T_{moyenne}$ (en S) & $3.09 \pm 0.02$ & $4.02 \pm 0.02$ & $5.07 \pm 0.02$ & $6.29 \pm 0.02$ & $7.42 \pm 0.02$\\
        \hline
    \end{tabular}
	\caption{Période moyenne $T_{moyenne}$ sur 10 mesures pour chaque distance $r_{masse}$}
	\label{table:mesure_D_dynamique}
\end{table}
\begin{figure}[h!]
    \begin{center}
        \includegraphics*[scale=1]{img/D_dynamique.png}
    \end{center}
    \caption{$\frac{T^2}{4\pi^2}$ en fonction de $r^2$}
    \label{fig:graphe_T}
\end{figure}

\textbf{Remarque:} Les incertitudes sur $T^2$ sont très faibles, et sont obtenues par la formule suivante:
\begin{equation}
    \frac{\Delta T^2}{T^2} = 2 \frac{\delta T}{T}
\end{equation}
Les incertitudes sur $r^2$ et qui suivent sur $D$ et $I_b$ sont obtenues par cette même méthode.

La régression linéaire nous donne les coefficients $a$ et $b$ avec les incertitudes associées:
\begin{equation}
    a = 17.1 \pm 0.2 s^2.m^{-2} \quad \quad \quad b = 0.159 \pm 0.004 s^2
\end{equation}

On en déduit alors les valeurs de $D$ et $I_b$:
\begin{align*}
    D = \frac{2m}{a} & = 24.8 \pm 0.3 mN.m \\
    I_b = \frac{2mb}{a} & = 3.9 \pm 0.2 g.m^2
\end{align*}


\break
\subsection{Période et frottements de l'air}
\label{section:periode}
Reprenons l'équation différentielle (\ref{eqn:equa_diff}):
\begin{equation}
    \ddot \theta + \frac{D}{I}\theta = 0 \Rightarrow \ddot \theta + \omega_0^2 \theta = 0, \quad \omega_0^2 = \frac{D}{I}
    \label{eqn:sans_frottements}
\end{equation}


En présence de frottements $\vec{f} = -k \vec{v}$, cette même équation prend la forme suivante:
\begin{equation}
    \ddot \theta + \frac{k r^2}{I} \dot \theta + \frac{D}{I}\theta = 0
\end{equation}

En effet, on a:
\begin{align}
    \vec{M}(\vec{f}) = \vec{OM} \wedge \vec{f} = r\ur \wedge -k r \dot \theta \ut = -k r^2 \dot \theta \uz
\end{align}

Pour simplifier, posons simplement que la force s'applique en bout de la barre, et on supposera donc un coefficient 
$\alpha = k R^2$, avec $R$ la demi-longueur de la barre.

On se place dans le cas d'une oscillation pseudo-période, donc $\lambda < \omega_0$. Nous avons alors:
\begin{equation}
    \ddot \theta + \frac{\alpha}{I} \dot \theta + \frac{D}{I}\theta = 0 \Rightarrow \ddot \theta + 2\lambda \dot \theta + \omega_0^2\theta = 0, \quad 2\lambda = \frac{\alpha}{I}, \omega = \sqrt{\omega_0^2 - \lambda^2} 
    \label{eqn:avec_frottements}
\end{equation}

On notera la solution sans frottements $\theta_{sf}(t)$ et celle avec frottements $\theta_{f}(t)$. On pose
les conditions initiales pour les deux résolutions comme $\theta(0) = \theta_0 \text{ et } \dot \theta(0) = 0$. La résolution des équations 
différentielles donnent alors:
\begin{align}
    \theta_{sf}(t) &= \theta_0 \cos(\omega_0 t), \quad  \\
    \theta_{f}(t) &= \theta_0 \exp(-\lambda t)\cos(\omega t) = A(t) \cos(\omega t)
\end{align}

On remarque qu'entre les deux expressions, seul un terme de décroissance exponentielle intervient pour rendre compte des frottements.
Ainsi, seul l'amplitude se voit décroître, mais la période reste inchangée.
On remarque également que l'angle de départ n'influence pas la période, mais uniquement l'amplitude de l'oscillation.

\subsection{Moment d'inertie d'une boule}
\label{boule}
Soit une boule de rayon $R$ et de masse volumique $\rho$ uniforme. On a:
\begin{align}
    I_{boule} = \int_{r=0}^{R}\int_{\theta=0}^{2\pi}\int_{\phi=0}^{\pi} \rho (x^2 + y^2) dxdy
\end{align}
On passe en coordonnées sphériques:
\begin{align*}
    x &= r \sin \theta \cos \phi \\
    y &= r \sin \theta \sin \phi \\
    z &= r \cos \theta \\
    dV &= r^2 sin \theta dr d\theta d\phi
\end{align*}

\break
Il vient alors que:
\begin{align*}
    I_{boule} & = \iiint_V \rho (x^2 + y^2) dxdy \\
    & = \rho \int_{r=0}^{R}\int_{\phi=0}^{2\pi}\int_{\theta=0}^{\pi} r^4 \sin^3 \theta dr d\theta d\phi \\
    & = \rho \int_{r=0}^{R} r^4 dr \int_{\phi=0}^{2\pi} d\phi \int_{\theta=0}^{\pi} \sin^3 \theta d\phi \\
    & = 2\pi \rho \left[ \frac{r^5}{5} \right]_0^R \int_{\theta=0}^{\pi} \sin^3 \theta d\theta \\
    & = \frac{4 \times 2 \pi \rho R^5}{3 \times 5 } \\
    & = \frac{2 m R^3}{5}
\end{align*}

\subsection{Détails calculs d'incertitudes}
\label{incertitude}

Soit $T_i$, la i-ème mesure sur les 10 mesures effectuées par solide, on calcule la moyenne et l'incertitude sur $T$ par:
\begin{align*}
    T &= \frac{1}{10} \sum_{i=1}^{10} T_i \\
    \delta T &= \sqrt{\frac{1}{10 \times 9} \sum_{i=1}^{10} (T_i - T)}
\end{align*}

De part l'incertitude sur le temps de réaction de l'expérimentateur, on multipliera $\delta T$ par un facteur $t = 1.38$ selon la loi de Student pour monter
l'incertitude de confiance de $68\%$ à $80\%$ pour être plus représentatif de la présence d'un temps de réaction de l'expérimentateur.

Enfin, pour comparer les intervalles de confiance entre la valeur théorique $x_{th}$ et la valeur expérimentale $x_{exp}$, on notera $\gamma$ le pourcentage
d'intervalle de $x_{exp}$ coupé par l'intervalle de $x_{th}$. On a alors:
\begin{equation}
    \gamma = \frac{\text{Largeur intervalle expérimental coupé par }x_{th}}{2\delta x_{exp} + 1} \times 100
\end{equation}


\end{document}