\documentclass[12pt]{article}
\usepackage[left=2cm, top=2cm, right=2cm, bottom=2cm]{geometry}
\usepackage[utf8]{inputenc}
\usepackage[T1]{fontenc}
\usepackage[french]{babel}
\usepackage{graphicx}
\usepackage{graphics}
\usepackage{amsmath}
\usepackage{tikz}
\usepackage{graphicx}
\usepackage{xcolor}
\usepackage{parskip}
\usepackage{physics}


\title{\textbf{Méthodes expérimentales} \\ TP 2: Collisions}
\author{MENARD Alexandre \\ VIEILLEDENT Florent}

\setlength{\parindent}{1cm}

\begin{document}
\maketitle

\section*{Introduction}

\newpage
\section{Champ magnétique généré par une seule bobine}
Dans un premier temps, on s'intéresse à modéliser le champ magnétique d'une seule bobine comportant 
$N = 154 spires$, de rayon $r = 20cm$ (A MODIFIER APRES MESURE).

\subsection{Théorie par la loi de Biot-Savart}
Dans une bobine de rayon $r$ parcourue par un courant $I$. On pose $P$, le point se trouvant sur la spire,
, $M$, le point où l'on souhaite mesurer le champ magnétique, et $O$, le centre de la bobine. On munira notre espace d'une base cylindrique vu
la symétrie du problème. On aura donc les trois vecteurs unitaires suivant : $(\vec{e_r}, \vec{e_\theta}, \vec{e_z})$.

\begin{equation}
    \vec{B}(\vec{r}) = \frac{\mu_0}{4 \pi} \oint_C \frac{I d\vec{l} \wedge (\vec{PM})}{PM^3}
\end{equation}

Sachant que la distance $PM$ est constante lorsque l'on parcoure la spire, $I$ est constant, on peut les sortir. De plus,
on découpe l'intégrale en deux en notant que $\vec{PM} = \vec{PO} + \vec{OM}$.

\begin{equation}
    \vec{B}(\vec{r}) = \frac{I \mu_0}{4 \pi PM^3} \left( \oint_C d\vec{l} \wedge \vec{PO} + \oint_C d\vec{l} \wedge \vec{OM} \right)
\end{equation}

On exprime les deux intégrales à part:
\begin{gather*}
    \oint_C d\vec{l} \wedge \vec{OM} = \oint_C rd\theta \vec{e_\theta} \wedge z\vec{e_z} = \oint_C rd\theta \vec{e_r} = \vec{0} \\
    \oint_C d\vec{l} \wedge \vec{PO} = \oint_C rd\theta \vec{e_\theta} \wedge r * (-\vec{e_r}) = \oint_C r^2 d\theta \vec{e_z} = 2 \pi r^2\vec{e_z}
\end{gather*}

On a finalement que:
\begin{equation}
    \vec{B}(\vec{r}) = \frac{I \mu_0 r^2}{2 PM^3} \vec{e_z}
\end{equation}

Cependant, on a $PM = \sqrt{r^2 + z^2}$ donc:
\begin{equation}
    \vec{B}(\vec{r}) = \frac{I \mu_0 r^2}{2 \sqrt{r^2 + z^2}^3} \vec{e_z}
\end{equation}

Lorsqu'on se place dans le centre de la bobine, c'est à dire $z = 0$, on a plus simplement que:
\begin{equation}
    \vec{B_0}(\vec{r}) = \frac{I \mu_0 r^2}{2r} \vec{e_z} = \frac{I \mu_0}{2r} \vec{e_z}
\end{equation}

Et dans le cas d'une bobine à $N$ spires, on a finalement que:
\begin{equation}
    \vec{B_0}(\vec{r}) = \frac{N I \mu_0}{2r} \vec{e_z}
\end{equation}

\subsection{Expérimentation}
Pour cette expérience, on positionne une bobine de $N = 154$ spires et de rayon $r = 20cm$ (A MESURER) au centre du banc de telle sorte à pouvoir réaliser
autant de mesures de chaque côté de cette dernière. Avec une source de courant, on alimente la bobine avec 
un courant $I = 2A$. A l'aide d'un ampèremètre, on mesure le courant réel parcourant notre bobine, on a $I_r = ?? A$.

Pour commencer, on souhaite déterminer l'influence du sens du courant sur le champ magnétique. Pour cela, on suit le protocole
suivant:

\begin{enumerate}
    \item A l'aide d'une boussole, on détermine la direction et le sens du champ magnétique $\vec{B}$, ainsi que les lignes de champs
    en déplaçant la boussole le long de l'axe de la bobine en s'éloignant. \label{etape:boussole}
    \item En prenant soin d'éteindre la source de courant, on inverse le courant et l'on allume de nouveau la source.
    \item On détermine de nouveau l'orientation du champ magnétique avec la boussole en suivant l'étape \ref{etape:boussole}. 
\end{enumerate}

En inversant le courant, on remarque que la boussole pointe dans le sens opposé, mais les lignes de champs restent identiques. On représente
les lignes de champs dans la figure suivante:

\begin{equation}
    \text{[INSERER FIGURE]}
\end{equation}

Ensuite, on détermine l'influence de la distance $z$ avec la bobine sur l'intensité du champ magnétique. Pour cela, avec une sonde de Hall, on mesure l'intensité 
du champ magnétique sur l'axe $Oz$, à différentes distances $z$, en faisant cela de chaque côté de la bobine.
On trace donc l'intensité de $\vec{B}$ selon la distance $z$ avec la bobine. 

\begin{figure}[h!]
    \begin{center}
        % \includegraphics[]{}
        \text{[GRAPHIQUE CHAMP MAGNETIQUE SELON OZ]}
    \end{center}
    \caption{Influence de la distance $z$ sur l'intensité du champ magnétique}
\end{figure}

Enfin, on mesure l'effet de l'intensité du courant sur l'intensité du champ magnétique. On positionne donc notre
sonde de Hall (la pointe) au centre de la bobine (ie $z=0m$). On règle la valeur du courant $I$ à différentes valeurs, et l'on mesure 
le courant réel parcourant notre bobine avec un ampèremètre et l'intensité du champ magnétique. On trace donc $\norm{\vec{B}}$ en fonction $I$.

\begin{figure}[h!]
    \begin{center}
        % \includegraphics[]{}
        \text{[GRAPHIQUE CHAMP MAGNETIQUE SELON I]}
    \end{center}
    \caption{Influence de l'intensité du courant sur l'intensité du champ magnétique}
\end{figure}

\newpage

\subsection{Modélisation}
Dans cette partie, nous allons comparer nos valeurs expérimentales à la théorie vue en tout premier lieu. Ainsi,
on comparera l'intensité du champ magnétique au centre de la bobine à ce que la théorie prédit, puis on comparera la courbe
de l'intensité du champ magnétique selon $z$ face à la l'intensité attendue selon la théorie, tout cela avec un courant $I$ fixe.

\newpage

\end{document}
