\documentclass[12pt]{article}
\usepackage[left=3cm, top=3cm, right=3cm, bottom=3cm]{geometry}
\usepackage[utf8]{inputenc}      % accents dans le source
\usepackage[T1]{fontenc}
\usepackage[french]{babel}
\usepackage{graphicx}
\usepackage{graphics}
\usepackage{amsmath}
\usepackage{tikz}
\usepackage{xcolor} 
\usepackage{mathtools}
\usepackage{parskip}
\usepackage{subcaption}
\usepackage[export]{adjustbox}
\usepackage{chemist}

\title{\textbf{TP1 Chimie des Solutions} \\ Dosage colorimétrique de l’aspirine dans un comprimé}
\author{MENARD Alexandre \\ VIEILLEDENT Florent}

\begin{document}
\maketitle

\section*{Introduction}

Dans ce travail pratique, on détermineras la masse d'acide acétylsalicylique dans un comprimé d'aspirine grâce à un titrage par de la soude. 
On commenceras par étalonner notre solution de soude par dosage pH-métrique et dosage colorimétrique avec de l'acide oxalique. Puis on effectueras les mêmes dosages mais avec l'acide acétylsalicylique et notre soude étalonnée. 
\newpage

\section{Étalonnage de la solution de soude}

Le but de cette première expérience est de déterminer précisément la concentration d'une solution pour l'utiliser par le suite pour titrer une autre solution. On dose notre solution par de l'acide oxalique en utilisant du phénolphtaléine comme colorant pour la colorimétrie. On effectue un dosage pH-métrique en même temps. 

	\subsection{Montage expérimental}
	
\begin{figure}[!h]
	\begin{center}
\includegraphics[scale=0.5]{Schéma_Titrage1.png}
\label{Figure1}
\caption{Schéma du montage expérimental du titrage de la soude par l'acide oxalique}
\end{center}
\end{figure}
	
	\subsection{Résultat}

	\subsection{Traitement des résultats}
	
	\subsection{Conclusion}
	
\section{Dosage de l'acide acétylsalicylique}
	\subsection{Montage expérimental}
	
	\subsection{Résultat}

	\subsection{Traitement des résultats}
	
	\subsection{Conclusion}

\end{document}