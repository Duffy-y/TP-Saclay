\documentclass[12pt]{article}
\usepackage[left=2cm, top=2cm, right=2cm, bottom=2cm]{geometry}
\usepackage[utf8]{inputenc}
\usepackage[T1]{fontenc}
\usepackage[french]{babel}
\usepackage{graphicx}
\usepackage{graphics}
\usepackage{amsmath}
\usepackage{tikz}
\usepackage{graphicx}
\usepackage{xcolor}
\usepackage{parskip}
\usepackage{physics}
\usepackage{tikz}


\title{\textbf{Méthodes expérimentales} \\ TP 4: Étude des propriétés thermodynamiques d'un gaz presque parfait}
\author{MENARD Alexandre \\ VIEILLEDENT Florent}

\setlength{\parindent}{1cm}

\begin{document}
\maketitle

\section*{Introduction}
Dans ce travail pratique, on cherchera à vérifier la loi des gaz parfaits. Le gaz utilisé sera l'air qu'on considère ici comme un gaz parfait. Dans un premier temps, on mesureras le pression d'un gaz en faisant varier le volume de ce gaz avec une température constante. On utiliseras pour cela une seringue et un pressiomètre Jeulin. 

On mesureras ensuite la pression d'un gaz à volume constant en faisant varier la température. On utilise pour ça une bouteille étanche et un manomètre. On utilise pour cela un gaz dans une seringue et un pressiomètre Jeulin. 





\newpage

\section{Relation entre le volume et la pression à température donnée}

Dans cette première expérience on cherche à vérifier que le produit PV est une constante. Pour cela, on calcule la pression d'un gaz pour différentes valeurs de volumes, sans changer la température. 	

\subsection{Expérimentation}

La seringue est placé initialement sur $15\, cm^3$. On relie ensuite la seringue au pressiomètre Jeulin gràce à un tuyau. On fait varier le volume de la seringue et on note la pression donnée par le pressiomètre. 

\begin{figure}[!h]
	\begin{center}
		\includegraphics[scale=0.3]{Schéma_seringue.png}
		\label{Schéma_seringue}
		\caption{Schéma de la première expérience avec la seringue et le pressiomètre}
	\end{center}
\end{figure}

[INSÉRER PHOTO PREMIÈRE EXPÉRIENCE]

On note $V_{seringue}$ le volume à l'intérieur de la seringue. On fait varier ce volume de $15\, cm^3$ à $60\, cm^3$ tout les $5\, cm^3$. La seringue est graduée tout les $cm^3$, donc notre incertitude sur le volume est $\delta V_{seringue}=0.5 \, cm^3$.  On note pour chaque volume la pression donnée par le pressiomètre. L'incertitude sur le pressiomètre est de $2\% $. On a donc $\delta P = 0.02*P$. On répète les mesures 3 fois pour avoir une meilleur estimation de l'incertitude. On note respectivement P1,P2 et P3 les mesures de pressions lors de la première, deuxième et troisième mesure. 

\begin{table}[h!]
	\begin{center}
		\begin{tabular}{|r|r|r|r|r|r|r|}	
\hline
 $V_{seringue} (cm^3)$ &  P1 (hPa) &  $\delta P1(hPa)$ &  P2 (hPa) &  $\delta P2(hPa)$ &  P3 (hPa) &  $\delta P3(hPa)$ \\
\hline
            15 &              1003 &            20 &              1050 &            21 &              1070 &            21 \\
            20 &               792 &            16 &               810 &            16 &               816 &            16 \\
            25 &               640 &            13 &               654 &            13 &               652 &            13 \\
            30 &               542 &            11 &               552 &            11 &               556 &            11 \\
            35 &               468 &             9 &               474 &             9 &               480 &            10 \\
            40 &               410 &             8 &               417 &             8 &               427 &             9 \\
            45 &               368 &             7 &               374 &             7 &               374 &             7 \\
            50 &               331 &             7 &               337 &             7 &               338 &             7 \\
            55 &               302 &             6 &               304 &             6 &               308 &             6 \\
            60 &               276 &             6 &               283 &             6 &               284 &             6 \\
\hline
		\end{tabular}
	\end{center}
\end{table}

\end{document}
